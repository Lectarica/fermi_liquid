\documentclass[11pt,a4j]{jarticle}
\usepackage{comment}
\usepackage{float}
\usepackage{color}
\usepackage{multicol}
\usepackage[dvipdfmx]{pict2e}
\usepackage{braket}
\usepackage{wrapfig}
\usepackage{graphicx}
\usepackage{bm}
\usepackage{url}
\usepackage{underscore}
\usepackage{colortbl}
\usepackage{tabularx}
\usepackage{fancyhdr}
\usepackage{ulem}
\usepackage{cite}
\usepackage{amsmath,amssymb,amsfonts}
\usepackage{braket}
\usepackage{algorithmic}
\usepackage{textcomp}
\usepackage{xcolor}
\usepackage[ipaex]{pxchfon}
\usepackage[top=30truemm,bottom=30truemm,left=25truemm,right=25truemm]{geometry}
\title{微視的フェルミ流体論}
\author{Shimura Koki}
\date{}
\begin{document}
\maketitle
\section{道具立て}
\subsection{前書き(ノーテーション確認)}
ランダウのフェルミ流体理論は準粒子描像を手掛かりに
現象論的な結果を導いてきたが、ダイアグラムを導入して
任意の次数に関する摂動展開の理論を展開させたのは
アブリコソフとカラトニコフであった。

\subsection{2粒子グリーン関数}
2粒子グリーン関数を、4個のハイゼンベルク演算子のT積を基底状態についての
平均
\begin{equation}
    K_{34,12}(\bm{x}_1t_1,\bm{x}_2t_2;\bm{x}^{'}_1t^{'}_1,\bm{x}^{'}_2t^{'}_2) =
     \langle T [\hat{\psi}_3 \hat{\psi}_4 \hat{\psi}^{\dagger}_1\hat{\psi}^{\dagger}_2]\rangle
\end{equation}
で定義する。最も低次の近似では、1粒子グリーン関数$G^{0}$の積の和に分解される。
\begin{equation}
    K_{34,12}^{0} = G_{31}^{(0)}G_{42}^{(0)} - G_{32}^{(0)}G_{41}^{(0)} 
\end{equation}
これはダイアグラムで表すと次のようになる。
以下、運動量表示による議論を行う。

\subsection{バーテックス関数}

\section{フェルミ流体の基礎的性質}

\section{LW理論}

\section{Ward恒等式}

\section{ランダウ理論との整合性}


\begin{thebibliography}{99}
    \bibitem{pitaevskii} リフシッツ・ピタエフスキー、『量子統計物理学』(岩波書店)
    \bibitem{luttinger-noziers}P. Nozières and J. M. Luttinger, Phys. Rev. $\bm{127}, 1423, 1431(1962)$
    \bibitem{luttinger-ward}J. M. Luttinger and J. C. Ward, Phys. Rev. $\bm{118}, 1417(1960)$
    \bibitem{luttinger-kohn}
\end{thebibliography}

\end{document}